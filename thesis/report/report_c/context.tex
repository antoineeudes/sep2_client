\chapter{Context}\label{ch:context}

IEEE 2030.5 is but one protocol designed for communication between end-user energy devices and electric utilities. 
Although 2030.5 is a modern protocol, it is not wholly new or original. Rather, it is the product of historically successful protocols designed with the same aim.
In this section we'll examine the predecessors to IEEE 2030.5, and their influence on the standard.

\section{Smart Energy Profile 1.x}
Developed by ZigBee Alliance, and published in 2008, Smart Energy Profile 1.x is a specification for an application-layer communication protocol between end-user energy devices and electric utilities. 
The specification called for the usage of the "ZigBee" communication protocol, based off the IEEE 802.15.4 specification for physical layer communication. \cite[]{ZigBeeSEP} \hfill \break
The specification was adopted by utilities worldwide, including the Southern California Edison Company, who purchased usage of the system for \$400 million USD. \cite[]{SEP1Article} \hfill \break
According to SunSpec in 2019, over 60 million smart meters are still deployed under ZigBee Smart Energy 1.x, with 550 certified SEP 1.x products. \cite[]{20305workshop}


\section{SunSpec Modbus}
Referenced in the specification as the foundations for IEEE 2030.5 is the Sunspec Alliance Inverter Control Model, which encompasses the SunSpec Modbus Protocol.
Sunspec Modbus is an extension of the Modbus communication protocol, also designed for end-user energy devices, and was published, yet not standardised, in 2010. The protocol set out to accomplish many of the same goals as IEEE 2030.5 does today. 
Tom Tansy, chairman of the SunSpec Alliance pins the goal of the protocol as to create a 'common language that all distributed energy component manufacturers could use to enable communication interoperability'.
\cite[]{SunspecModbusArticle}

\section{IEEE 2030}
IEEE 2030 was a guide, published in 2011, to help standardise smart grid communication and interoperability, and describe how potential solutions could be evaluated. A major goal of these potential communication protocols is that, by their nature of existing in the end-user environment, they were to prioritise the security of all data stored and transmitted, such that communication between electric utilities cannot be intercepted, monitored or tampered by unauthorised users.

Furthermore, protocols used were to ensure that an electric grid denial of service cannot be brought about by attacks on smart grid communication infrastructure \cite{2030Security} \cite[]{2030}. 


\section{Producing IEEE 2030.5}

In the interest of interoperability with a future standard, the SunSpec alliance donated their SunSpec Modbus protocol to form IEEE 2030.5. Simultaneously, ZigBee Alliance was looking to develop Smart Energy Profile 2.0, which would use TCP/IP.
At this point the SunSpec Alliance formed a partnership with ZigBee Alliance, and IEEE 2030.5 was created as a TCP/IP communication protocol that is both SEP 2.0, and interoperable with SunSpec Modbus \cite[]{SunspecModbusArticle}. 

Likewise, IEEE 2030.5 works to improve the security of smart grid communication protocols, and the guiding principles put forward by IEEE 2030. 

The extensibility of the protocol was also considered in it's design. IEEE 2030.5 provides a standard method for extending it's functionality, such that legislative, state-specific, and proprietary extensions of the standard can be developed whilst retaining the protocol's core design. Examples of this include the later discussed CSIP and CSIP-AUS, where both allow for clients \& servers to be deployed under a different model from that describe in the specification. CSIP-AUS extends the possible structured data that can be communicated between clients \& servers to better fit the requirements of electric utilities in Australia.

With IEEE 2030.5 extending and combining these existing, widely used, protocols, we're reassured it actually solves the problems faced by utilities and smart grid device manufacturers alike, and wasn't created in a vacuum, unaware of real world requirements, or the needs of device manufacturers and electric utilities.


