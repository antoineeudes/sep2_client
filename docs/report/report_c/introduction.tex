\chapter{Introduction}\label{ch:intro}
Society faces a growing need for reliable, sustainable and affordable electricity. One such way we have attempted to address this problem is via the invention of the 'smart grid', an electric grid assisted by computers, where by communication is enabled between electric utilities and end-users via a computer network, such as the Internet.

The portion of the smart grid existing in the end-user environment are end-user energy devices.
This category of end-user energy devices further encompasses the category of "Distributed Energy Resources", devices that deliver AC power to be consumed in the residence and/or exported back to the electric grid. Examples include solar inverters, household solar batteries, and biomass generators \cite{IEEE2030.5}. 

Through the use of distributed energy resources, fossil fuel based energy generation can be more readily replaced with clean and renewable energy, the need for which is of growing importance as we seek to address the threat of global climate change. Of great importance to the success of Distributed Energy Resources is their integration, and ongoing management as part of the broader electric grid.

End-user energy devices may require communication with electric utilities for the purpose of managing electric supply and demand, monitoring usage, and ensuring end-users are compensated, and charged, for their energy supply and demand, respectively.

2030.5 is an IEEE standardised communication protocol purpose built for securely integrating end-user energy devices, and therefore Distributed Energy Resources into the wider electric grid. Since it's inception in 2013, the protocol has seen both minor and major revisions, and has seen unanimous adoption by DNSPs in both Australia, and the United States of America - in the Australian Common Smart Inverter Protocol (CSIP-AUS) \cite{CSIPAus} and Californian Smart Inverter Protocol \cite{CalAdoption}, respectively.

Thus, the goal of this thesis is to implement a safe, secure, reliable and performant framework for developing IEEE 2030.5 clients for use in the smart grid ecosystem. 