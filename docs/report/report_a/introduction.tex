\chapter{Introduction}\label{ch:intro}
In recent history, society has faced by a growing need for reliable, sustainable and affordable electricity.
Modern technology is used to address this problem by way of the 'smart grid', an electric grid assisted by computers, to enable communication and control between utilities and individuals on the network.

On this smart grid are 'end-user energy devices', devices existing in the end-user energy environment.
This category of end-user energy devices encompasses, what are referred to as 'Distributed Energy Resources',
devices that deliver AC power to be consumed in the residence and/or the grid, such as solar inverters, and batteries. \cite{IEEE2030.5}

Through the use of distributed energy resources, fossil fuel based energy generation can be replaced with clean, 
renewable energy, the need for which is of growing importance as humanity seeks to address the threat of global climate change.

However, of great importance to the success of Distributed Energy Resources is their management, and integration, into the wider electric grid.
End-user energy devices require communication with electric utilities for the purpose of managing electricity supply and demand, monitoring usage, 
and ensuring end-users are compensated for their energy contributions, as to name a few.

The 2030.5 Protocol is an IEEE standardised communication protocol purpose built for integrating end-user energy devices, and therefore Distributed Energy Resources, into the wider electric grid.
The protocol has seen unanimous adoption, or proposed adoption, by DNSPs in Australia.\cite{DOEAdoption}

Thus, the goal of this thesis is to implement the IEEE 2030.5 standard for use in end-user energy devices.
% An 2030.5 Client refers to the portion of this protocol that is implemented on the side of an end-user energy device.


