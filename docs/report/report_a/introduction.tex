\chapter{Introduction}\label{ch:intro}
In recent history, society has faced by a growing need for reliable, sustainable and affordable electricity.
A modern invention built to address this are end user energy devices, acting as distributed energy resources. 
Distributed Energy Resources are small-scale renewable energy generation and storage systems existing in the end-user environment, primarily residential dwellings.

Through the use of distributed energy resources, fossil fuel based energy generation can be replaced with clean, 
renewable energy, the need for which is of growing importance as humanity seeks to address the threat of global climate change.
Furthermore, Distributed Energy Resources work to improve the resiliency and reliability of energy distribution, by decentralizing energy generation and storage.

However, of great importance to the success of Distributed Energy Resources is their management, and integration, into the wider electric grid.
End user energy devices require management by electric utilities for the purpose of managing electricity supply and demand, monitoring usage, 
and ensuring end-users are compensated for their energy contributions.

The 2030.5 Protocol is an IEEE standardised communication protocol purpose built for integrating end user energy devices, and therefore Distributed Energy Resources, into the wider electric grid.
It does this by specifying, what, and how, these end-user energy devices should communicate with servers hosted by the electric utility. 
An IEEE 2030.5 Client refers to the portion of this protocol that is implemented on he side of an end-user energy device.


