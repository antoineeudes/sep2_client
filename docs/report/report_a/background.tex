\chapter{Background}\label{ch:background}

\section{High-Level Architecture}
The IEEE 2030.5 protocol is a REST API, and as such, adopts a client-server model.

Transmitted between clients and servers are 'Resources', all of which are defined in a standardised schema. 
Despite the client-server model, the IEEE 2030.5 specification purposefully does not make distinctions between clients and servers, as to avoid resources having differing behaviours on each. Rather, a server simply exposes resources to clients, and clients retrieve, update, create and delete resources on servers.
A client communicates with, at most, one server, and a server communicates with many clients.

Being the product of existing technologies, the IEEE 2030.5 resources cover a wide range of applications, and as such, the specification logically groups resources into discrete 'function sets', of which there are twenty-five. 
Device manufacturers or electric utilities implementing IEEE 2030.5 need only communicate resources from function sets relevant to the purpose of the device. 

The specification defines two methods by which clients retrieve resources from server. The first of which, polling, has a client contact a server, requesting updates, on a timed interval. 
The second of which has clients 'subscribe' to a resource, where by a server will then send a notification to all subscribed clients when that resource is updated.
\cite{IEEE2030.5}

\section{Protocol Design}
Resources are transmitted between clients and servers using HTTP/1.1, over TCP/IP using TLS.
As a result, the protocol employs the usual HTTP request methods of GET, POST, PUT and DELETE for retrieving, updating, creating and deleting resources, respectively.

The specification requires that TLS certificates be signed, and all encryption done, using ECC cipher suites, with the ability to use RSA cipher suites as a fallback.

The standardised resource schema is defined using XSD, as all transmitted resources are represented using either XML, or EXI, with the HTTP/1.1 \texttt{Content-Type} header set to \texttt{sep+xml} or \texttt{sep-exi}, respectively.

In order to connect to servers, clients must be able to resolve hostnames to IP addresses using DNS. Similarly, the specification permits the ability for clients to discover servers on a local network using DNS-SD.

\section{Usage}
The IEEE 2030.5 function sets cover a wide range of applications and uses, aiming to support as many end-user energy devices as possible. A subset of these possible use cases are as follows:
\begin{itemize}
    \item (Smart) Electricity Meters can use the 'Metering' function set to 'exchange commodity measurement information' using 2030.5 resources. \cite{IEEE2030.5}
    \item Electric Vehicle chargers may wish to have their power usage behaviour influenced by resources belonging to the 'Demand Load and Response Control' function set.
    \item Solar Inverters may use the 'Distributed Energy Resources' function set such that their energy output into the wider grid can be controlled by the utility.
\end{itemize}
